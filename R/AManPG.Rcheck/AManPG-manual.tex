\nonstopmode{}
\documentclass[a4paper]{book}
\usepackage[times,inconsolata,hyper]{Rd}
\usepackage{makeidx}
\usepackage[utf8]{inputenc} % @SET ENCODING@
% \usepackage{graphicx} % @USE GRAPHICX@
\makeindex{}
\begin{document}
\chapter*{}
\begin{center}
{\textbf{\huge Package `AManPG'}}
\par\bigskip{\large \today}
\end{center}
\inputencoding{utf8}
\ifthenelse{\boolean{Rd@use@hyper}}{\hypersetup{pdftitle = {AManPG: What the Package Does (Title Case)}}}{}\begin{description}
\raggedright{}
\item[Type]\AsIs{Package}
\item[Title]\AsIs{What the Package Does (Title Case)}
\item[Version]\AsIs{0.1.0}
\item[Author]\AsIs{Who wrote it}
\item[Maintainer]\AsIs{The package maintainer }\email{yourself@somewhere.net}\AsIs{}
\item[Description]\AsIs{More about what it does (maybe more than one line)
Use four spaces when indenting paragraphs within the Description.}
\item[License]\AsIs{What license is it under?}
\item[Encoding]\AsIs{UTF-8}
\item[NeedsCompilation]\AsIs{no}
\end{description}
\Rdcontents{\R{} topics documented:}
\inputencoding{utf8}
\HeaderA{hello}{Hello, World!}{hello}
%
\begin{Description}\relax
Prints 'Hello, world!'.
\end{Description}
%
\begin{Usage}
\begin{verbatim}
hello()
\end{verbatim}
\end{Usage}
%
\begin{Examples}
\begin{ExampleCode}
hello()
\end{ExampleCode}
\end{Examples}
\inputencoding{utf8}
\HeaderA{prox\_l1}{Proximal L1 Mapping}{prox.Rul.l1}
%
\begin{Description}\relax
Honestly I don't know what this does yet haha
\end{Description}
%
\begin{Usage}
\begin{verbatim}
prox_l1(x)
\end{verbatim}
\end{Usage}
%
\begin{Arguments}
\begin{ldescription}
\item[\code{x}] 


\end{ldescription}
\end{Arguments}
%
\begin{Examples}
\begin{ExampleCode}
##---- Should be DIRECTLY executable !! ----
##-- ==>  Define data, use random,
##--	or do  help(data=index)  for the standard data sets.

## The function is currently defined as
function (x)
{
  }
\end{ExampleCode}
\end{Examples}
\inputencoding{utf8}
\HeaderA{spca\_amanpg}{Alternativing Manifold Proximal Gradient Method for Sparse PCA}{spca.Rul.amanpg}
%
\begin{Description}\relax

Executes sparse PCA using the AManPG algorithm.
\end{Description}
%
\begin{Usage}
\begin{verbatim}
spca_amanpg(B, mu, lambda, n, type, maxiter, tol, X0, Y0, F_palm)
\end{verbatim}
\end{Usage}
%
\begin{Arguments}
\begin{ldescription}
\item[\code{b}] Data or covariance matrix to be analyzed
\item[\code{mu}] Good question!
\item[\code{n}] Number of columns in x, y
\item[\code{x0}] Initial x value for gradient algorithm
\item[\code{y0}] Initial y value for gradient algorithm
\item[\code{type}] 
0 indicates \code{b} is data matrix, 1 indicates \code{b} is covariance matrix, default is 0

\item[\code{maxiter}] Maximum number of iterations prior to termination of algorithm, default is 1e4
\item[\code{tol}] Tolerance level for detecting convergence, default is 1e-5
\item[\code{f\_palm}] Good question! I can see it's a convergence condition tho
\item[\code{verbose}] Logical where TRUE displays progress during execution, default is FALSE
\end{ldescription}
\end{Arguments}
%
\begin{Value}
\begin{ldescription}
\item[\code{iter}] Number of iterations used in the algorithm
\item[\code{f\_amanpg}] Final value returned
\item[\code{sparsity}] Ratio of all values in returned matrix with zeroes
\item[\code{time}] Seconds used during execution
\item[\code{x}] Final x matrix
\item[\code{y\_man}] Final y matrix of principal components
\end{ldescription}
\end{Value}
%
\begin{References}\relax
Shixiang Chen, Shiqian Ma, Lingzhou Xue and Hui Zou. An Alternating Manifold Proximal Gradient Method for Sparse PCA and Sparse CCA. 2019.
\end{References}
%
\begin{Examples}
\begin{ExampleCode}
##---- Should be DIRECTLY executable !! ----
##-- ==>  Define data, use random,
##--	or do  help(data=index)  for the standard data sets.

## The function is currently defined as
function (x)
{
  }
\end{ExampleCode}
\end{Examples}
\inputencoding{utf8}
\HeaderA{svd.econ}{Economy-Size Singular Value Decomposition}{svd.econ}
%
\begin{Description}\relax
Produces an economy-size decomposition of \eqn{m}{}-by-\eqn{n}{} matrix \eqn{x}{}, similar to the functionality in MATLAB.
\end{Description}
%
\begin{Usage}
\begin{verbatim}
svd.econ(x, type=1)
\end{verbatim}
\end{Usage}
%
\begin{Arguments}
\begin{ldescription}
\item[\code{x}] Matrix to be decomposed into \eqn{UDV}{}
\item[\code{type}] If 0, will only consider the \code{m > n} case. If 1, will consider \code{m > n} and \code{m < n}
\end{ldescription}
\end{Arguments}
%
\begin{Details}\relax
If \code{m > n}, only the first \eqn{n}{} columns of \eqn{U}{} are computed, and \eqn{D}{} is of length \eqn{n}{}.

If \code{m < n}, only the first \eqn{m}{} columns of \eqn{V}{} are computed, and \eqn{D}{} is of length \eqn{m}{}.

If \code{m == n}, the base \code{svd} function will be executed.

Note that for \code{type = 0}, only the condition \code{m > n} will perform an economy-size decomposition. For \code{m <= n}, the base \code{svd} function is run.
\end{Details}
%
\begin{Author}\relax
Justin Huang
\end{Author}
%
\begin{References}\relax
https://www.mathworks.com/help/matlab/ref/double.svd.html
\end{References}
%
\begin{SeeAlso}\relax
\code{\LinkA{svd}{svd}}
\end{SeeAlso}
%
\begin{Examples}
\begin{ExampleCode}
set.seed(10)
x <- matrix(rnorm(3 * 4), 3, 4)
udv <- svd.econ(x)

udv.0 <- svd.econ(x, 0)
\end{ExampleCode}
\end{Examples}
\printindex{}
\end{document}
